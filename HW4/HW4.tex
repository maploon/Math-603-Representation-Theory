\documentclass[10pt,a4paper]{article}

%%%%%%%%% Packages %%%%%%%%
\usepackage{amsmath}
\usepackage{amsthm}
\usepackage{amsfonts}
\usepackage{amssymb}
\usepackage{latexsym}
\usepackage{epsfig}
\usepackage{graphicx}
\usepackage{fancyhdr}
\usepackage{titling}
\usepackage{lipsum}
\usepackage{tikz}
\usepackage{float}
\usepackage[a4paper, total={7in, 9.5in}]{geometry}
\usepackage[linesnumbered,boxruled,commentsnumbered]{algorithm2e}
\usepackage[matrix,tips,graph,curve]{xy}

%%%%%%%%%%%%%%%%%%%%%%%%%%%%%%%%%%
\makeatletter
\@addtoreset{equation}{section}
\makeatother

%%%%%%%%%%%%%%%%%%%%%%%%%%%%%%%%%%

\renewcommand{\theequation}{\thesection.\arabic{equation}}

\theoremstyle{plain}
\newtheorem{theorem}[equation]{Theorem}
\newtheorem{corollary}[equation]{Corollary}
\newtheorem{lemma}[equation]{Lemma}
\newtheorem{proposition}[equation]{Proposition}
\newtheorem{conjecture}[equation]{Conjecture}
\newtheorem{fact}[equation]{Fact}
\newtheorem{facts}[equation]{Facts}
\newtheorem*{theoremA}{Theorem A}
\newtheorem*{theoremB}{Theorem B}
\newtheorem*{theoremC}{Theorem C}
\newtheorem*{theoremD}{Theorem D}
\newtheorem*{theoremE}{Theorem E}
\newtheorem*{theoremF}{Theorem F}
\newtheorem*{theoremG}{Theorem G}
\newtheorem*{theoremH}{Theorem H}

\theoremstyle{definition}
\newtheorem{definition}[equation]{Definition}
\newtheorem{definitions}[equation]{Definitions}

\newtheorem{remark}[equation]{Remark}
\newtheorem{remarks}[equation]{Remarks}
\newtheorem{exercise}[equation]{Exercise}
\newtheorem{example}[equation]{Example}
\newtheorem{examples}[equation]{Examples}
\newtheorem{notation}[equation]{Notation}
\newtheorem{question}[equation]{Question}
\newtheorem{assumption}[equation]{Assumption}
\newtheorem*{claim}{Claim}
\newtheorem{problem}{Problem}
\newtheorem*{problem*}{Problem}


\theoremstyle{remark}
\newtheorem{answer}{Answer}
\newtheorem*{answer*}{Answer}

%%%%%% letters %%%%
\newcommand{\fA}{\mathfrak{A}}
\newcommand{\fB}{\mathfrak{B}}
\newcommand{\fC}{\mathfrak{C}}
\newcommand{\fD}{\mathfrak{D}}
\newcommand{\fE}{\mathfrak{E}}
\newcommand{\fF}{\mathfrak{F}}
\newcommand{\fG}{\mathfrak{G}}
\newcommand{\fH}{\mathfrak{H}}
\newcommand{\fI}{\mathfrak{I}}
\newcommand{\fJ}{\mathfrak{J}}
\newcommand{\fK}{\mathfrak{K}}
\newcommand{\fL}{\mathfrak{L}}
\newcommand{\fM}{\mathfrak{M}}
\newcommand{\fN}{\mathfrak{N}}
\newcommand{\fO}{\mathfrak{O}}
\newcommand{\fP}{\mathfrak{P}}
\newcommand{\fQ}{\mathfrak{Q}}
\newcommand{\fR}{\mathfrak{R}}
\newcommand{\fS}{\mathfrak{S}}
\newcommand{\fT}{\mathfrak{T}}
\newcommand{\fU}{\mathfrak{U}}
\newcommand{\fV}{\mathfrak{V}}
\newcommand{\fW}{\mathfrak{W}}
\newcommand{\fX}{\mathfrak{X}}
\newcommand{\fY}{\mathfrak{Y}}
\newcommand{\fZ}{\mathfrak{Z}}
%%%%%%%%%%%%%%%%%%%%%%%%%%%%%%
\newcommand{\fa}{\mathfrak{a}}
\newcommand{\fb}{\mathfrak{b}}
\newcommand{\fc}{\mathfrak{c}}
\newcommand{\fd}{\mathfrak{d}}
\newcommand{\fe}{\mathfrak{e}}
\newcommand{\ff}{\mathfrak{f}}
\newcommand{\fg}{\mathfrak{g}}
\newcommand{\fh}{\mathfrak{h}}
\newcommand{\ffi}{\mathfrak{i}} %%\fi is defined
\newcommand{\fj}{\mathfrak{j}}
\newcommand{\fk}{\mathfrak{k}}
\newcommand{\fl}{\mathfrak{l}}
\newcommand{\fm}{\mathfrak{m}}
\newcommand{\fn}{\mathfrak{n}}
\newcommand{\fo}{\mathfrak{o}}
\newcommand{\fp}{\mathfrak{p}}
\newcommand{\fq}{\mathfrak{q}}
\newcommand{\fr}{\mathfrak{r}}
\newcommand{\fs}{\mathfrak{s}}
\newcommand{\ft}{\mathfrak{t}}
\newcommand{\fu}{\mathfrak{u}}
\newcommand{\fv}{\mathfrak{v}}
\newcommand{\fw}{\mathfrak{w}}
\newcommand{\fx}{\mathfrak{x}}
\newcommand{\fy}{\mathfrak{y}}
\newcommand{\fz}{\mathfrak{z}}
%%%%%%%%%%%%%%%%%%%%%%%%%%%%%%%
\newcommand{\sA}{\mathcal{A}\,}
\newcommand{\sB}{\mathcal{B}\,}
\newcommand{\sC}{\mathcal{C}}
\newcommand{\sD}{\mathcal{D}\,}
\newcommand{\sE}{\mathcal{E}\,}
\newcommand{\sF}{\mathcal{F}\,}
\newcommand{\sG}{\mathcal{G}\,}
\newcommand{\sH}{\mathcal{H}}
\newcommand{\sI}{\mathcal{I}\,}
\newcommand{\sJ}{\mathcal{J}\,}
\newcommand{\sK}{\mathcal{K}\,}
\newcommand{\sL}{\mathcal{L}\,}
\newcommand{\sM}{\mathcal{M}\,}
\newcommand{\sN}{\mathcal{N}}
\newcommand{\sO}{\mathcal{O}}
\newcommand{\sP}{\mathcal{P}\,}
\newcommand{\sQ}{\mathcal{Q}\,}
\newcommand{\sR}{\mathcal{R}}
\newcommand{\sS}{\mathcal{S}}
\newcommand{\sT}{\mathcal{T}\,}
\newcommand{\sU}{\mathcal{U}\,}
\newcommand{\sV}{\mathcal{V}\,}
\newcommand{\sW}{\mathcal{W}\,}
\newcommand{\sX}{\mathcal{X}\,}
\newcommand{\sY}{\mathcal{Y}\,}
\newcommand{\sZ}{\mathcal{Z}\,}
%%%%%%%%%%%%%%%%%%%%%%%%%%%%%%%
\newcommand{\IA}{\mathbb{A}}
\newcommand{\IB}{\mathbb{B}}
\newcommand{\IC}{\mathbb{C}}
\newcommand{\ID}{\mathbb{D}}
\newcommand{\IE}{\mathbb{E}}
\newcommand{\IF}{\mathbb{F}}
\newcommand{\IG}{\mathbb{G}}
\newcommand{\IH}{\mathbb{H}}
\newcommand{\II}{\mathbb{I}}
\newcommand{\IK}{\mathbb{K}}
\newcommand{\IL}{\mathbb{L}}
\newcommand{\IM}{\mathbb{M}}
\newcommand{\IN}{\mathbb{N}}
\newcommand{\IO}{\mathbb{O}}
\newcommand{\IP}{\mathbb{P}}
\newcommand{\IQ}{\mathbb{Q}}
\newcommand{\IR}{\mathbb{R}}
\newcommand{\IS}{\mathbb{S}}
\newcommand{\IT}{\mathbb{T}}
\newcommand{\IU}{\mathbb{U}}
\newcommand{\IV}{\mathbb{V}}
\newcommand{\IW}{\mathbb{W}}
\newcommand{\IX}{\mathbb{X}}
\newcommand{\IY}{\mathbb{Y}}
\newcommand{\IZ}{\mathbb{Z}}
%%%%%%%%%%%%%%%%%%%%%%%%%%%%%%
\newcommand{\tA}{\mathrm {A}}
\newcommand{\tB}{\mathrm {B}}
\newcommand{\tC}{\mathrm {C}}
\newcommand{\tD}{\mathrm {D}}
\newcommand{\tE}{\mathrm {E}}
\newcommand{\tF}{\mathrm {F}}
\newcommand{\tG}{\mathrm {G}}
\newcommand{\tH}{\mathrm {H}}
\newcommand{\tI}{\mathrm {I}}
\newcommand{\tJ}{\mathrm {J}}
\newcommand{\tK}{\mathrm {K}}
\newcommand{\tL}{\mathrm {L}}
\newcommand{\tM}{\mathrm {M}}
\newcommand{\tN}{\mathrm {N}}
\newcommand{\tO}{\mathrm {O}}
\newcommand{\tP}{\mathrm {P}}
\newcommand{\tQ}{\mathrm {Q}}
\newcommand{\tR}{\mathrm {R}}
\newcommand{\tS}{\mathrm {S}}
\newcommand{\tT}{\mathrm {T}}
\newcommand{\tU}{\mathrm {U}}
\newcommand{\tV}{\mathrm {V}}
\newcommand{\tW}{\mathrm {W}}
\newcommand{\tX}{\mathrm {X}}
\newcommand{\tY}{\mathrm {Y}}
\newcommand{\tZ}{\mathrm {Z}}
%%%%%%% macros %%%%%

%% my definitions %%%

\newcommand{\End}{\mathrm{End}}
\newcommand{\tr}{\mathrm{tr}}
%\newcommand{\ind}{\mathrm{ind}}

\renewcommand{\index}{\mathrm{index \,}}
\newcommand{\Hom}{\mathrm{Hom}}
\newcommand{\Aut}{\mathrm{Aut}}
\newcommand{\Trace}{\mathrm{Trace}\,}
\newcommand{\Res}{\mathrm{Res}\,}
\newcommand{\rank}{\mathrm{rank}}
%\renewcommand{\dim}{\mathrm{dim}}

\renewcommand{\deg}{\mathrm{deg}}
\newcommand{\spin}{\rm Spin}
\newcommand{\Spin}{\rm Spin}
\newcommand{\erfc}{\rm erfc\,}
\newcommand{\sgn}{\rm sgn\,}
\newcommand{\Spec}{\rm Spec\,}
\newcommand{\diag}{\rm diag\,}
\newcommand{\Fix}{\mathrm{Fix}}
\newcommand{\Ker}{\mathrm{Ker \,}}
\newcommand{\Coker}{\mathrm{Coker \,}}
\newcommand{\Sym}{\mathrm{Sym \,}}
\newcommand{\Hess}{\mathrm{Hess \,}}
\newcommand{\grad}{\mathrm{grad \,}}
\newcommand{\Center}{\mathrm{Center}}
\newcommand{\Lie}{\mathrm{Lie}}
\newcommand{\ch}{\rm ch} % Chern Character
\newcommand{\rk}{\rm rk}
\newcommand{\sign}{\rm sign}
\renewcommand\dim{{\rm dim\,}}
\renewcommand\det{{\rm det\,}}
\newcommand{\dimKrull}{{\rm Krulldim\,}}
\newcommand\Rep{\mathrm{Rep}}
\newcommand\Hilb{\mathrm{Hilb}}
\newcommand\vol{\mathrm{vol}}
\newcommand\QED{\hfill $\Box$} %{\bf QED}}
\newcommand\Pf{\nonintend{\em Proof. }}
\newcommand\reals{{\mathbb R}}
\newcommand\complexes{{\mathbb C}}
\renewcommand\i{\sqrt{-1}}
\renewcommand\Re{\mathrm Re}
\renewcommand\Im{\mathrm Im}
\newcommand\integers{{\mathbb Z}}
\newcommand\quaternions{{\mathbb H}}
\newcommand\iso{{\cong}}
\newcommand\tensor{{\otimes}}
\newcommand\Tensor{{\bigotimes}}
\newcommand\union{\bigcup}
\newcommand\onehalf{\frac{1}{2}}
%\newcommand\Sym[1]{{Sym^{#1}(\complexes^2)}}
\newcommand\lie[1]{{\mathfrak #1}}
\renewcommand\fk{\mathfrak{K}}
\newcommand\smooth{\mathcal{C}^{\infty}}
\newcommand\trivial{{\mathbb I}}
\newcommand\widebar{\overline}

%%%%%Delimiters%%%%

\newcommand{\<}{\langle}
\renewcommand{\>}{\rangle}

%\renewcommand{\(}{\left(}
%\renewcommand{\)}{\right)}


%%%% Different kind of derivatives %%%%%
\newcommand{\delbar}{\bar{\partial}}
\newcommand{\pdu}{\frac{\partial}{\partial u}}
%\newcommand{\pd}[1][2]{\frac{\partial #1}{\partial #2}}

%%%%% Arrows %%%%%
\newcommand{\ra}{\rightarrow}                   % right arrow
%\newcommand{\lra}{\longrightarrow}              % long right arrow
%\renewcommand{\la}{\leftarrow}                    % left arrow
%\newcommand{\lla}{\longleftarrow}               % long left arrow
%\newcommand{\ua}{\uparrow}                     % long up arrow
%\newcommand{\na}{\nearrow}                      %  NE arrow
%\newcommand{\llra}[1]{\stackrel{#1}{\lra}}      % labeled long right arrow
%\newcommand{\llla}[1]{\stackrel{#1}{\lla}}      % labeled long left arrow
%\newcommand{\lua}[1]{\stackrel{#1}{\ua}}      % labeled  up arrow
%\newcommand{\lna}[1]{\stackrel{#1}{\na}}      % labeled long NE arrow

\newcommand{\into}{\hookrightarrow}
\newcommand{\tto}{\longmapsto}
\def\llra{\longleftrightarrow}

\def\d/{/\mspace{-6.0mu}/}
\newcommand{\git}[3]{#1\d/_{\mspace{-4.0mu}#2}#3}
\newcommand\zetahilb{\zeta_{{\text{Hilb}}}}
\def\Fy{\sF \mspace{-3.0mu} \cdot \mspace{-3.0mu} y}
\def\tv{\tilde{v}}
\def\tw{\tilde{w}}
\def\wt{\widetilde}
\def\wtilde{\widetilde}
\def\what{\widehat}

%%%%%%%%%%%%%%%%%%% Mark's definitions %%%%%%%%%%%%%%%%%%%%

\newcommand{\frakg}{\mbox{\frakturfont g}}
\newcommand{\frakk}{\mbox{\frakturfont k}}
\newcommand{\frakp}{\mbox{\frakturfont p}}
\newcommand{\q}{\mbox{\frakturfont q}}
\newcommand{\frakn}{\mbox{\frakturfont n}}
\newcommand{\frakv}{\mbox{\frakturfont v}}
\newcommand{\fraku}{\mbox{\frakturfont u}}
\newcommand{\frakh}{\mbox{\frakturfont h}}
\newcommand{\frakm}{\mbox{\frakturfont m}}
\newcommand{\frakt}{\mbox{\frakturfont t}}
\newcommand{\G}{\Gamma}
\newcommand{\g}{\gamma}
\newcommand{\fraka}{\mbox{\frakturfont a}}
\newcommand{\db}{\bar{\partial}}
\newcommand{\dbs}{\bar{\partial}^*}
\newcommand{\p}{\partial}

%%%%%%%%%%%%% new definitions for the positive mass paper %%%%%%%%%

\newcommand{\sperp}{{\scriptscriptstyle \perp}}

%%%%%%%%%%%%% My definitions %%%%%%%%%%%%%%%%%%%%

\newcommand{\isom}{\cong}
\newcommand{\norm}[1]{\left\lVert#1\right\rVert}

%%%%%%%% Page Layout %%%%%%%%

\linespread{1.065}

\newcommand{\subtitle}[1]{%
  \posttitle{%
    \par\end{center}
    \begin{center}\large#1\end{center}
    \vskip0.5em}%
}


%%%%%%%%%%%%%%%%%%%%%%%%%%%%%%%%%%%%%%%%%%%%%%%%%%%%%%%%%%%%%%%

%
\begin{document}
%

%%%%%%%% Title %%%%%%%
\title{Math 603}
\subtitle{Homework 4}
\author{Feng Gui}
\date{\today}

%%%%%%%% Headers and Footers %%%%%%%%

\fancypagestyle{plain}{%
  \renewcommand{\headrulewidth}{0pt}
  \fancyhf{}%
  \rfoot{PAGE \thepage}
}

\pagestyle{plain}

\makeatletter
\let\runlhead\@author
\let\runrhead\@title
\makeatother

\renewcommand{\headrulewidth}{1.5pt}
\lhead{Homework 4} %%%% Use \runlhead to put author on left header. Manually put subtitle if there is one
\chead{}
\rhead{\runrhead}

\lfoot{}
\cfoot{}

%%%%%%%% Make Title %%%%%%%%

\maketitle

%%%%%%%% Body %%%%%%%%

\section{Symmetry}
\label{sec:Symmetry}

\begin{problem*} \bfseries 2
\end{problem*}

\begin{answer*} \hfill

(ii) Suppose $\norm{f_X(t)} \leq b$ for some $b$. Then $\norm{f_X(t)-Id} \leq \norm{f_X(t)} + \norm{Id} \leq b+b_0 $ is bounded. Since $X\not = 0$, $\norm{X} = c$ for some $c$. We then need to find an incredibly large $a$, so that
\begin{align*}
\norm{f_X(a)-Id} &= \norm{aX+\frac{1}{2}a^2X^2+\cdots+\frac{1}{k!}a^kX^k} \\
&\geq \norm{\frac{1}{k!}a^kX^k} - \norm{aX}-\norm{\frac{1}{2}a^2X^2}-\cdots - \norm{\frac{1}{(k-1)!}a^{k-1}X^{k-1}}\\
&\geq \frac{1}{k!}(ac)^k - ac - \frac{1}{2}(ac)^2 - \cdots - \frac{1}{(k-1)!}(ac)^{k-1}
\end{align*}
Since $ac$ can be arbitrarily large and $k$ is fixed for an strictly upper triangular matrix $X$, this is unbounded. The contradiction shows that $f_X(t)$ is not bounded. \\

(iii) We have that $\exp(tX) = diag(\exp{tx_{11}},...,\exp{tx_{nn}})$. Since whether a set is bounded or not does not depend on the norm we choose, we choose the norm of $X$ to be $\left(\sum_{i,j} \norm{x_{ij}}\right)^{\frac{1}{2}}$. Then apparently when $x_{11},...,x_{nn}$ are purely imaginary, $\exp(tX)$ is bounded by $\sqrt{n}$. Notice that $\norm{\exp(tX)} = \left(\sum_{i} \exp(t\mathrm{Re}(x_{ii}))\right)^\frac{1}{2}$.
If any diagonal entry of $X$ has real part, then this is certainly not bounded.\\

(vii) If $X = SYS^{-1}$ with $Y$ being a diagonal matrix of purely imaginray numbers, then $\exp(tX) = S\exp(tY)S^{-1}$ is certainly bounded by the proof above. On the other hand, let $X = PJP^{-1}$ with $J$ being its jordan canonical form. We have $J = D+N$ where $D$ is diagonal and $N$ is nilpotent and $DN = ND$. Then $\exp(tX) = S\exp(tD)p(tN)S^{-1}$ for some polynomial $p$. If this is bounded, then $\exp(tD)$ and $p(tN)$ is bounded. By question (ii), $p(tN)$ being bounded means $N = 0$. $\exp(tD)$ is bounded, so it has only purely imaginary eigenvalues. Then $X$ is similar to a diagonal matrix that has only purely imaginary eigenvalues. \\

(viii) The closure of the image is compact if and only if it is bounded since a subspace of Euclidean space is compact if and only if it is closed and bounded. By criterion above, $f_X(t)$ is bounded only when $X$ is similar to a diagonal matrix with only purely imaginary eigenvalues. Notice that such a matrix is skew-hermitian and all skew-hermitain matrix are diagonalizble and all eigenvalues are imaginary by spectral theorem (for skew-hermitian matrices). Therefore $X$ must be similar to a skew-hermitian matrix.\\
\end{answer*}

\section{Exercise Sheet 2}
\label{sec:Exercise Sheet 2}

\begin{problem*} \bfseries 1
\end{problem*}

\begin{answer*} \hfill

(a) First we show that $\exp(tX)$ actually lives in $GL_2(\IR)$. This is because $\det \exp(tX) = \exp(\tr(tX)) \not = 0$. Hence it is invertible. Also, since the exponential map is $\sC^\infty$ with respect to $t$, the map keeps the $\sC^\infty$ manifold structure. Now we just need to show that it is a homomorphism between groups.
Since $X$ commute with itself, this follows from:
\[r(s+t) = \exp(sX+tX) = \exp(sX)\exp(tX)\]

Therefore, $r_X$ is a homomorphism of Lie group, $(\IR, +)$. \\
\hfill

(b)(i) Since $X^2 = 0$, $$r(t) = \exp(tX) = 1+tX = \begin{pmatrix} 1 & t \\ 0 & 1\end{pmatrix}$$ So there's a subrepresentation of $r$, $W = \mathrm{Span}\{\begin{pmatrix}1\\0\end{pmatrix}\}$. It is obvious that for any $t$, $r(t)w\in W$ for any $w\in W$. Therefore it is not irreducible.\\

(b)(ii) Notice that $\begin{pmatrix} 1 & t \\ 0 & 1\end{pmatrix}$ only has one eigenvalue associate with one eigenvector. Therefore, there's no other $\IR$-stable subspace of $V$ that is a subrepresentation. It follows that $V$ can't be the direct sum of several subrepresentation. Hence $r$ is not completely reducible. \\

(b)(iii) $W$ as described above is $\IR\times\{0\} \cong \IR$, therefore $\IR^2/W \cong \IR$. Notice that $r(t)$ is the identity of $W$ and the quotient representation is the identity $GL(\IR^2/W)$. The direct sum of subrepresentation and the quotient representation is
\[r\oplus\bar{r}(t)(w,v) = (v,w) \in W\oplus\IR^2/W \isom\IR^2\].

Therefore the direct sum of two representations is isomorphic to $r'(t) = Id\in GL_2(\IR)$. \\

(b)(iv) Since $W\oplus\IR^2/W$ is isomorphic to $\IR^2$, we identiy $\rho(t) = r\oplus\bar{r}(t) = Id\in GL(W\oplus\IR^2/W) \isom GL_2{\IR}$.  If $\rho$ is isomorphic to $r$ then there exists $T:\IR^2\ra \IR^2$ such that $\rho(t)\circ T = T \circ r(t)$. Then $r(t)$ has to be diagonalizable. However it is not similar to any diagonal matrix as the eigenspace is not the who space. Therefore these two representation are not isomorphic.\\

(c)(i) Since $X$ has two different eigenvalues, $X$ is diagonalizable. Therefore let $X = SJS^{-1}$ with $J$ being a diagonal matrix. Then $$r(t) = \exp(tX) = \exp(tSJS^{-1}) = S\exp(tJ)S^{-1}$$
Since $J$ is a diagonal matrix, $\exp(tJ)$ is also a diagonal matrix. Therefore $\exp(tX)$ has the same eigenvectors as $\exp(tJ)$ and thus it must have two eigenvectors. Suppose $v$ is one eigenvector of $\exp(J)$, then since $t^kJ^kv = t^k\lambda^k v$, it is also an eigenvector of $\exp(tJ)$. So the eigenvectors of $\exp(tX)$ doesn't depend on $t$. Since each eigenvector forms a $\IR$-stable subspace, the representation is not irreducible. \\

(c)(ii) Suppose the eigenvectors for $r(t) = \exp(tX)$ are $v$ and $w$. In, fact $v$ and $w$ are also eigenvectors of $X$. Then $V = \{sv,\, \forall s\in \IR\}$ and $W = \{sw,\, \forall s\in \IR\}$ are two stable subspaces with $W\oplus V = \IR^2$. Therefore, $r$ is completely reducible. \\

(c)(iii) Let $r_1:\IR \ra GL(V)$ and $r_2:\IR \ra GL(W)$ be two representations: $r_1(t)v = r(t)v = \lambda_1 v$ and $r_2(t)v = r(t)w = \lambda_2 w$. These are the irreducible representations since $\dim V = \dim W = 1$. Then let $\rho(t)(v,w) = r_1\oplus r_2(t)(v,w) = (\lambda_1 v, \lambda_2 w)$ be the direct sum of $r_1$ and $r_2$. Let $v'$ and $w'$ be the eigenvectors of $r(t)$ which is basically $v$ and $w$ in $\IR^2$.
We use different notation since here we are interpreting $v$ and $w$ as basis for vector space $V$ and $W$. Since $v'$ and $w'$ forms a basis for $\IR^2$, There's a unique linear transformation $T:V\oplus W$ such that $T(v) = v'$ and $T(w) = w'$. Then we have $r(t)v' = T\circ\rho(t)\circ T^{-1}v'$ and $r(t)w' = T\circ\rho(t)\circ T^{-1}w'$. Therefore this explicitly constructs an isomorphism. \\

(d)(i) Suppose $\exp(tX)v = \lambda(t)v$ for all $t\in \IR$. Then we have $\frac{d}{dt} \exp(tX) v = \lambda'(t)v$. However, since $\frac{d}{dt}\exp(tX) = X\exp(tX)$, we have $\frac{d}{dt} \exp(tX) v = \lambda(t)Xv = \lambda'(t)v$. But we know that $X$ doesn't have any eigenvector. Therefore $\exp(tX)$ doesn't have a eigenvector that is not depend on $t$. So there won't be $\IR$-stable subspace. Hence the representation is irreducible.\\
\end{answer*}


\section{Tensor Products of Modules}
\label{sec:Tensor Products of Modules}

\begin{problem*} \bfseries 6.5
\end{problem*}

\begin{answer*} \hfill

(i) Let $v$ be the basis for $V$. Then $v\wedge v = 0$. Therefore $\Lambda^k(V) = \{0\}$ for $k>1$. Thus
\[\Lambda^\bullet(V) \isom \Lambda^0(V)\oplus \Lambda^1(V) \isom k \oplus V \isom V \]

(ii) Similarly, $\Lambda^3(V) = \{0\}$ as the wedge of three basis $e_1$ and $e_2$ results to 0. Thus Thus $\Lambda^0(V) = k$ has $1$ which is the identity of field $k$, $\Lambda^1(V)$ has basis $e_1$ and $e_2$, and $\Lambda^2(V)$ has basis $e_1\wedge e_2$ . \\

(iii) The ideal generated by $e_1\wedge e_2\wedge\cdots\wedge e_n$ where $e_1,e_2,...,e_n$ form a basis of $V$
\end{answer*}



\begin{problem*} 6.6
\end{problem*}

\begin{answer*} \hfill

(ii) Let $e_1, e_2$ be the natural basis for $V$. Then $T^2(V)$ has basis $e_1\otimes e_1,e_1\otimes e_2, e_2\otimes e_1, e_2\otimes e_2$. Then the matrix of $f^{\otimes 2}$ is
\[\begin{pmatrix} a^2 & ab & ab & b^2\\ 0 &  ad & 0 & bd \\ 0 & 0 & ad & bd \\0 & 0 & 0 & d^2
 \end{pmatrix}\]

(iii) For $Sym^2(V)$:
\[f^{\otimes 2} = \begin{pmatrix} a^2 & ab & b^2 \\ 0 & ad & 2bd \\ 0 & 0 & d^2
 \end{pmatrix}\]

For $\Lambda^2(V)$
\[f^{\otimes 2} = bd\]

\end{answer*}



\begin{problem*} 6.7
\end{problem*}

\begin{answer*} \hfill

(ii) Since the wedge product of any $n+1$ basis vectors of $V=k^n$ is equal to 0, $\Lambda^k(V) = \{0\}$ for $k>n$. Then for $k\leq n$. $\Lambda^k(V)$ has $\begin{pmatrix}n \\ k \end{pmatrix}$ basis vectors since the wedge of any $k$ basis vectors of $V$ forms a basis vector of $\Lambda^k(V)$. Therefore the dimension is $$\sum_{k=0}^n \begin{pmatrix} n \\ k \end{pmatrix} = 2^n$$

\end{answer*}


\end{document}
