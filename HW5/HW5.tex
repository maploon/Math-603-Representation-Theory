\documentclass[10pt,a4paper]{article}

%%%%%%%%% Packages %%%%%%%%
\usepackage{amsmath}
\usepackage{amsthm}
\usepackage{amsfonts}
\usepackage{amssymb}
\usepackage{latexsym}
\usepackage{epsfig}
\usepackage{graphicx}
\usepackage{fancyhdr}
\usepackage{titling}
\usepackage{lipsum}
\usepackage{tikz}
\usepackage{float}
\usepackage[a4paper, total={7in, 9.5in}]{geometry}
\usepackage[linesnumbered,boxruled,commentsnumbered]{algorithm2e}
\usepackage[matrix,tips,graph,curve]{xy}

%%%%%%%%%%%%%%%%%%%%%%%%%%%%%%%%%%
\makeatletter
\@addtoreset{equation}{section}
\makeatother

%%%%%%%%%%%%%%%%%%%%%%%%%%%%%%%%%%

\renewcommand{\theequation}{\thesection.\arabic{equation}}

\theoremstyle{plain}
\newtheorem{theorem}[equation]{Theorem}
\newtheorem{corollary}[equation]{Corollary}
\newtheorem{lemma}[equation]{Lemma}
\newtheorem{proposition}[equation]{Proposition}
\newtheorem{conjecture}[equation]{Conjecture}
\newtheorem{fact}[equation]{Fact}
\newtheorem{facts}[equation]{Facts}
\newtheorem*{theoremA}{Theorem A}
\newtheorem*{theoremB}{Theorem B}
\newtheorem*{theoremC}{Theorem C}
\newtheorem*{theoremD}{Theorem D}
\newtheorem*{theoremE}{Theorem E}
\newtheorem*{theoremF}{Theorem F}
\newtheorem*{theoremG}{Theorem G}
\newtheorem*{theoremH}{Theorem H}

\theoremstyle{definition}
\newtheorem{definition}[equation]{Definition}
\newtheorem{definitions}[equation]{Definitions}

\newtheorem{remark}[equation]{Remark}
\newtheorem{remarks}[equation]{Remarks}
\newtheorem{exercise}[equation]{Exercise}
\newtheorem{example}[equation]{Example}
\newtheorem{examples}[equation]{Examples}
\newtheorem{notation}[equation]{Notation}
\newtheorem{question}[equation]{Question}
\newtheorem{assumption}[equation]{Assumption}
\newtheorem*{claim}{Claim}
\newtheorem{problem}{Problem}
\newtheorem*{problem*}{Problem}


\theoremstyle{remark}
\newtheorem{answer}{Answer}
\newtheorem*{answer*}{Answer}

%%%%%% letters %%%%
\newcommand{\fA}{\mathfrak{A}}
\newcommand{\fB}{\mathfrak{B}}
\newcommand{\fC}{\mathfrak{C}}
\newcommand{\fD}{\mathfrak{D}}
\newcommand{\fE}{\mathfrak{E}}
\newcommand{\fF}{\mathfrak{F}}
\newcommand{\fG}{\mathfrak{G}}
\newcommand{\fH}{\mathfrak{H}}
\newcommand{\fI}{\mathfrak{I}}
\newcommand{\fJ}{\mathfrak{J}}
\newcommand{\fK}{\mathfrak{K}}
\newcommand{\fL}{\mathfrak{L}}
\newcommand{\fM}{\mathfrak{M}}
\newcommand{\fN}{\mathfrak{N}}
\newcommand{\fO}{\mathfrak{O}}
\newcommand{\fP}{\mathfrak{P}}
\newcommand{\fQ}{\mathfrak{Q}}
\newcommand{\fR}{\mathfrak{R}}
\newcommand{\fS}{\mathfrak{S}}
\newcommand{\fT}{\mathfrak{T}}
\newcommand{\fU}{\mathfrak{U}}
\newcommand{\fV}{\mathfrak{V}}
\newcommand{\fW}{\mathfrak{W}}
\newcommand{\fX}{\mathfrak{X}}
\newcommand{\fY}{\mathfrak{Y}}
\newcommand{\fZ}{\mathfrak{Z}}
%%%%%%%%%%%%%%%%%%%%%%%%%%%%%%
\newcommand{\fa}{\mathfrak{a}}
\newcommand{\fb}{\mathfrak{b}}
\newcommand{\fc}{\mathfrak{c}}
\newcommand{\fd}{\mathfrak{d}}
\newcommand{\fe}{\mathfrak{e}}
\newcommand{\ff}{\mathfrak{f}}
\newcommand{\fg}{\mathfrak{g}}
\newcommand{\fh}{\mathfrak{h}}
\newcommand{\ffi}{\mathfrak{i}} %%\fi is defined
\newcommand{\fj}{\mathfrak{j}}
\newcommand{\fk}{\mathfrak{k}}
\newcommand{\fl}{\mathfrak{l}}
\newcommand{\fm}{\mathfrak{m}}
\newcommand{\fn}{\mathfrak{n}}
\newcommand{\fo}{\mathfrak{o}}
\newcommand{\fp}{\mathfrak{p}}
\newcommand{\fq}{\mathfrak{q}}
\newcommand{\fr}{\mathfrak{r}}
\newcommand{\fs}{\mathfrak{s}}
\newcommand{\ft}{\mathfrak{t}}
\newcommand{\fu}{\mathfrak{u}}
\newcommand{\fv}{\mathfrak{v}}
\newcommand{\fw}{\mathfrak{w}}
\newcommand{\fx}{\mathfrak{x}}
\newcommand{\fy}{\mathfrak{y}}
\newcommand{\fz}{\mathfrak{z}}
%%%%%%%%%%%%%%%%%%%%%%%%%%%%%%%
\newcommand{\sA}{\mathcal{A}\,}
\newcommand{\sB}{\mathcal{B}\,}
\newcommand{\sC}{\mathcal{C}}
\newcommand{\sD}{\mathcal{D}\,}
\newcommand{\sE}{\mathcal{E}\,}
\newcommand{\sF}{\mathcal{F}\,}
\newcommand{\sG}{\mathcal{G}\,}
\newcommand{\sH}{\mathcal{H}}
\newcommand{\sI}{\mathcal{I}\,}
\newcommand{\sJ}{\mathcal{J}\,}
\newcommand{\sK}{\mathcal{K}\,}
\newcommand{\sL}{\mathcal{L}\,}
\newcommand{\sM}{\mathcal{M}\,}
\newcommand{\sN}{\mathcal{N}}
\newcommand{\sO}{\mathcal{O}}
\newcommand{\sP}{\mathcal{P}\,}
\newcommand{\sQ}{\mathcal{Q}\,}
\newcommand{\sR}{\mathcal{R}}
\newcommand{\sS}{\mathcal{S}}
\newcommand{\sT}{\mathcal{T}\,}
\newcommand{\sU}{\mathcal{U}\,}
\newcommand{\sV}{\mathcal{V}\,}
\newcommand{\sW}{\mathcal{W}\,}
\newcommand{\sX}{\mathcal{X}\,}
\newcommand{\sY}{\mathcal{Y}\,}
\newcommand{\sZ}{\mathcal{Z}\,}
%%%%%%%%%%%%%%%%%%%%%%%%%%%%%%%
\newcommand{\IA}{\mathbb{A}}
\newcommand{\IB}{\mathbb{B}}
\newcommand{\IC}{\mathbb{C}}
\newcommand{\ID}{\mathbb{D}}
\newcommand{\IE}{\mathbb{E}}
\newcommand{\IF}{\mathbb{F}}
\newcommand{\IG}{\mathbb{G}}
\newcommand{\IH}{\mathbb{H}}
\newcommand{\II}{\mathbb{I}}
\newcommand{\IK}{\mathbb{K}}
\newcommand{\IL}{\mathbb{L}}
\newcommand{\IM}{\mathbb{M}}
\newcommand{\IN}{\mathbb{N}}
\newcommand{\IO}{\mathbb{O}}
\newcommand{\IP}{\mathbb{P}}
\newcommand{\IQ}{\mathbb{Q}}
\newcommand{\IR}{\mathbb{R}}
\newcommand{\IS}{\mathbb{S}}
\newcommand{\IT}{\mathbb{T}}
\newcommand{\IU}{\mathbb{U}}
\newcommand{\IV}{\mathbb{V}}
\newcommand{\IW}{\mathbb{W}}
\newcommand{\IX}{\mathbb{X}}
\newcommand{\IY}{\mathbb{Y}}
\newcommand{\IZ}{\mathbb{Z}}
%%%%%%%%%%%%%%%%%%%%%%%%%%%%%%
\newcommand{\tA}{\mathrm {A}}
\newcommand{\tB}{\mathrm {B}}
\newcommand{\tC}{\mathrm {C}}
\newcommand{\tD}{\mathrm {D}}
\newcommand{\tE}{\mathrm {E}}
\newcommand{\tF}{\mathrm {F}}
\newcommand{\tG}{\mathrm {G}}
\newcommand{\tH}{\mathrm {H}}
\newcommand{\tI}{\mathrm {I}}
\newcommand{\tJ}{\mathrm {J}}
\newcommand{\tK}{\mathrm {K}}
\newcommand{\tL}{\mathrm {L}}
\newcommand{\tM}{\mathrm {M}}
\newcommand{\tN}{\mathrm {N}}
\newcommand{\tO}{\mathrm {O}}
\newcommand{\tP}{\mathrm {P}}
\newcommand{\tQ}{\mathrm {Q}}
\newcommand{\tR}{\mathrm {R}}
\newcommand{\tS}{\mathrm {S}}
\newcommand{\tT}{\mathrm {T}}
\newcommand{\tU}{\mathrm {U}}
\newcommand{\tV}{\mathrm {V}}
\newcommand{\tW}{\mathrm {W}}
\newcommand{\tX}{\mathrm {X}}
\newcommand{\tY}{\mathrm {Y}}
\newcommand{\tZ}{\mathrm {Z}}
%%%%%%% macros %%%%%

%% my definitions %%%

\newcommand{\End}{\mathrm{End}}
\newcommand{\tr}{\mathrm{tr}}
%\newcommand{\ind}{\mathrm{ind}}

\renewcommand{\index}{\mathrm{index \,}}
\newcommand{\Hom}{\mathrm{Hom}}
\newcommand{\Aut}{\mathrm{Aut}}
\newcommand{\Trace}{\mathrm{Trace}\,}
\newcommand{\Res}{\mathrm{Res}\,}
\newcommand{\rank}{\mathrm{rank}}
%\renewcommand{\dim}{\mathrm{dim}}

\renewcommand{\deg}{\mathrm{deg}}
\newcommand{\spin}{\rm Spin}
\newcommand{\Spin}{\rm Spin}
\newcommand{\erfc}{\rm erfc\,}
\newcommand{\sgn}{\rm sgn\,}
\newcommand{\Spec}{\rm Spec\,}
\newcommand{\diag}{\rm diag\,}
\newcommand{\Fix}{\mathrm{Fix}}
\newcommand{\Ker}{\mathrm{Ker \,}}
\newcommand{\Coker}{\mathrm{Coker \,}}
\newcommand{\Sym}{\mathrm{Sym \,}}
\newcommand{\Hess}{\mathrm{Hess \,}}
\newcommand{\grad}{\mathrm{grad \,}}
\newcommand{\Center}{\mathrm{Center}}
\newcommand{\Lie}{\mathrm{Lie}}
\newcommand{\ch}{\rm ch} % Chern Character
\newcommand{\rk}{\rm rk}
\newcommand{\sign}{\rm sign}
\renewcommand\dim{{\rm dim\,}}
\renewcommand\det{{\rm det\,}}
\newcommand{\dimKrull}{{\rm Krulldim\,}}
\newcommand\Rep{\mathrm{Rep}}
\newcommand\Hilb{\mathrm{Hilb}}
\newcommand\vol{\mathrm{vol}}
\newcommand\QED{\hfill $\Box$} %{\bf QED}}
\newcommand\Pf{\nonintend{\em Proof. }}
\newcommand\reals{{\mathbb R}}
\newcommand\complexes{{\mathbb C}}
\renewcommand\i{\sqrt{-1}}
\renewcommand\Re{\mathrm Re}
\renewcommand\Im{\mathrm Im}
\newcommand\integers{{\mathbb Z}}
\newcommand\quaternions{{\mathbb H}}
\newcommand\iso{{\cong}}
\newcommand\tensor{{\otimes}}
\newcommand\Tensor{{\bigotimes}}
\newcommand\union{\bigcup}
\newcommand\onehalf{\frac{1}{2}}
%\newcommand\Sym[1]{{Sym^{#1}(\complexes^2)}}
\newcommand\lie[1]{{\mathfrak #1}}
\renewcommand\fk{\mathfrak{K}}
\newcommand\smooth{\mathcal{C}^{\infty}}
\newcommand\trivial{{\mathbb I}}
\newcommand\widebar{\overline}

%%%%%Delimiters%%%%

\newcommand{\<}{\langle}
\renewcommand{\>}{\rangle}

%\renewcommand{\(}{\left(}
%\renewcommand{\)}{\right)}


%%%% Different kind of derivatives %%%%%
\newcommand{\delbar}{\bar{\partial}}
\newcommand{\pdu}{\frac{\partial}{\partial u}}
%\newcommand{\pd}[1][2]{\frac{\partial #1}{\partial #2}}

%%%%% Arrows %%%%%
\newcommand{\ra}{\rightarrow}                   % right arrow
%\newcommand{\lra}{\longrightarrow}              % long right arrow
%\renewcommand{\la}{\leftarrow}                    % left arrow
%\newcommand{\lla}{\longleftarrow}               % long left arrow
%\newcommand{\ua}{\uparrow}                     % long up arrow
%\newcommand{\na}{\nearrow}                      %  NE arrow
%\newcommand{\llra}[1]{\stackrel{#1}{\lra}}      % labeled long right arrow
%\newcommand{\llla}[1]{\stackrel{#1}{\lla}}      % labeled long left arrow
%\newcommand{\lua}[1]{\stackrel{#1}{\ua}}      % labeled  up arrow
%\newcommand{\lna}[1]{\stackrel{#1}{\na}}      % labeled long NE arrow

\newcommand{\into}{\hookrightarrow}
\newcommand{\tto}{\longmapsto}
\def\llra{\longleftrightarrow}

\def\d/{/\mspace{-6.0mu}/}
\newcommand{\git}[3]{#1\d/_{\mspace{-4.0mu}#2}#3}
\newcommand\zetahilb{\zeta_{{\text{Hilb}}}}
\def\Fy{\sF \mspace{-3.0mu} \cdot \mspace{-3.0mu} y}
\def\tv{\tilde{v}}
\def\tw{\tilde{w}}
\def\wt{\widetilde}
\def\wtilde{\widetilde}
\def\what{\widehat}

%%%%%%%%%%%%%%%%%%% Mark's definitions %%%%%%%%%%%%%%%%%%%%

\newcommand{\frakg}{\mbox{\frakturfont g}}
\newcommand{\frakk}{\mbox{\frakturfont k}}
\newcommand{\frakp}{\mbox{\frakturfont p}}
\newcommand{\q}{\mbox{\frakturfont q}}
\newcommand{\frakn}{\mbox{\frakturfont n}}
\newcommand{\frakv}{\mbox{\frakturfont v}}
\newcommand{\fraku}{\mbox{\frakturfont u}}
\newcommand{\frakh}{\mbox{\frakturfont h}}
\newcommand{\frakm}{\mbox{\frakturfont m}}
\newcommand{\frakt}{\mbox{\frakturfont t}}
\newcommand{\G}{\Gamma}
\newcommand{\g}{\gamma}
\newcommand{\fraka}{\mbox{\frakturfont a}}
\newcommand{\db}{\bar{\partial}}
\newcommand{\dbs}{\bar{\partial}^*}
\newcommand{\p}{\partial}

%%%%%%%%%%%%% new definitions for the positive mass paper %%%%%%%%%

\newcommand{\sperp}{{\scriptscriptstyle \perp}}

%%%%%%%%%%%%% My definitions %%%%%%%%%%%%%%%%%

\newcommand{\isom}{\cong}
\newcommand{\td}{\tilde}


%%%%%%%% Page Layout %%%%%%%%

\linespread{1.065}

\newcommand{\subtitle}[1]{%
  \posttitle{%
    \par\end{center}
    \begin{center}\large#1\end{center}
    \vskip0.5em}%
}


%%%%%%%%%%%%%%%%%%%%%%%%%%%%%%%%%%%%%%%%%%%%%%%%%%%%%%%%%%%%%%%

%
\begin{document}
%

%%%%%%%% Title %%%%%%%
\title{Math 603 - Representation Theory}
\subtitle{Homework 5}
\author{Feng Gui}
\date{\today}

%%%%%%%% Headers and Footers %%%%%%%%

\fancypagestyle{plain}{%
  \renewcommand{\headrulewidth}{0pt}
  \fancyhf{}%
  \rfoot{PAGE \thepage}
}

\pagestyle{plain}

\makeatletter
\let\runlhead\@author
\let\runrhead\@title
\makeatother

\renewcommand{\headrulewidth}{1.5pt}
\lhead{\runlhead} %%%% Use \runlhead to put author on left header. Manually put subtitle if there is one
\chead{}
\rhead{\runrhead}

\lfoot{}
\cfoot{}

%%%%%%%% Make Title %%%%%%%%

\maketitle

%%%%%%%% Body %%%%%%%%

\section*{Exercise 3}
\label{sec:Exercise 3}


\begin{problem*}1
\end{problem*}

\begin{answer*} \hfill

(i) Checked.

(ii) The norm is $z\bar{z}+w\bar{w}$, which is a non-negative real number. So the image is $\IR^+\cup \{0\}$.

(iii) If $det(h) = z\bar{z}+w\bar{w} \not = 0$. Then let the inverse of $\begin{pmatrix}z & w \\ -\bar{w} & \bar{z} \end{pmatrix}$ is
\[\frac{1}{z\bar{z}+w\bar{w}}\begin{pmatrix} \bar{z} & -w \\ \bar{w} & z \end{pmatrix}\]

(iv) $\mathbf{i}^2 = \mathbf{j}^2 = \mathbf{k}^2 = (-1, 0) = -1$ and $\mathbf{ij} = \mathbf{k}, \mathbf{ji} = (0, -i) = -\mathbf{k}$.

(v) The conjugate of $\mathbf{i}$, $\mathbf{j}$, $\mathbf{k}$ are $-\mathbf{i}$, $-\mathbf{j}$,  $-\mathbf{k}$ respectively.

(vi) $h^*h= det(h)I$ where $I$ is the identity of $GL_2(\IC)$.

(vii) $\{1,\mathbf{i},\mathbf{j},\mathbf{k}\}$.\\

\end{answer*}


\begin{problem*} 2
\end{problem*}

\begin{answer*} \hfill

(i) Reduced norm $n(z,w) = z\bar{z}+w\bar{w} = a^2+b^2+c^2+d^2$.

(ii) According to (i), the reduced norm equals to $1$ means $a^2+b^2+c^2+d^2 = 1$. So it is diffeomorphic to unit $S^3$ in $\IR^4$. That also means it is diffeomorphic to $S^3$ in general.

(iii) Yes. $h^* = h^{-1}$ in this case.

(iv) Yes. It is a closed subgroup of $GL_2(\IC)$. It is closed topologically since $1$ is closed in $\IR^\times$ and $det$ is a continuous map. Therefore the $\IH_1$ forms a Lie group.

(v) They are the same. Both are the unitary group of $GL_1(\IH)$. \\
\end{answer*}

\begin{problem*}3
\end{problem*}

\begin{answer*} \hfill

(i) Suppose we have $A = \begin{pmatrix} z_1 & z_2 \\ z_3 & z_4 \end{pmatrix}$. Then we can use the equation $\bar{A}^T A = 1$ and $det(A) = 1$ to solve $z_4$ and $z_3$ in terms of $z_1$ and $z_2$. We have that $z_4 = \bar{z_1}$ and $z_3 = -\bar{z_2}$. This shows that $SU(2)$ is actually the Hamiltonian quaternions of reduced norm 1, i.e. $SU(2) = \IH_1$.

(ii) As shown in problem 2, since $SU(2) = \IH_1$, it is diffeomorphic to $S^3$.

(iii) No. Because the image $det$ map of $U(2)$ is $U(1)$ thich is not isomorphic to $\IR^\times$ which is the image of $det$ map of $\IH$. \\

\end{answer*}

\begin{problem*} 4
\end{problem*}

\begin{answer*} \hfill

(i) Clearly trace maps $h = \begin{pmatrix}z & w\\ -\bar{w} & \bar{z}\end{pmatrix}$ to a real number $z+\bar{z} = 2\mathbf{Re}(z)$. It is symmetric since $\tr(AB) = \sum_{i,j} a_{ij}b{ji} = \sum_{i,j} b_{ji}a_{ij} = \tr(BA)$. It is also bilinear since $\tr(aA) = a\tr(A)$ and $A(aB) = (aA)B$ for a real number $a$.
Therefore it is a symmetric bilinear form.

(ii) $\tr(AB) = -x_1y_1-x_2y_2-x_3y_3$.

(iii) Yes. Notice that $\{1, \mathbf{i}, \mathbf{j}, \mathbf{k}\}$ is a basis for $\IH$ and $\mathbf{i},\, \mathbf{j}, \, \mathbf{k}$ has 0 trace. So $\II$ is spanned by $\{\mathbf{i},\mathbf{j},\mathbf{k}\}$. Since it is a $\IR$ vector space. Therefore $GL(\II)$ has entries in $\IR$
and it is isomorphic to $GL_3(\IR)$. Also notice that $\tr$ respect to the basis $\{\mathbf{i},\mathbf{j},\mathbf{k}\}$ is just the natural inner product in $\IR^3$ with a negative sign. So $\{A\in GL(\II) | (Ah_1,Ah_2) = (h_1,h_2)\}$ is just the orthonormoal group in $GL(\II) \isom GL_3(\IR)$. And if we require $det(A) = 1$, it is just the special orthogonal group of $GL_(\II)$.
Therefore $SO(3)\isom G$.

(iv) Check two condition of group action:
(a) $e*l = ele^{-1} = ele = l$, where $e$ is the identity of $H^\times \subset GL_2(\IC)$.
(b) $h_1*(h_2*l) = h_1*(h_2lh_2^{-1}) = h_1(h_2lh_2^{-1})h_1^{-1} = (h_1h_2)l(h_1h_2)^{-1} = (h_1h_2)*l$

Therefore, it is an action.

(v) Note that for $h\in \IH_1$ and $l_1,l_2\in \II$, we have that $(h*l_1,h*l_2) = \tr(hl_1h^{-1}hl_2h^{-1}) = \tr(hl_1l_2h^{-1}) = (hl_1l_2, h^{-1}) = (h^{-1},hl_1l_2) = \tr(l_1l_2) = (l_1,l_2)$. We also know that $det(h) = 1$ for any $h\in \IH_1$.
Therefore for any $h$, we have a unique $A\in G$ such that $h*l = Al$ for any $l\in \II$. The mapping $h$ to $A$ is a Lie group homomorphism. We can also see this as conjugation homomophism $H^\times\ra Aut(\IH)$ and it fixes $\II$.

(vi) $\IH_1$ has dimension $3$ and $G\isom SO(3)$ has dimension $3$ as well.

(vii) Since both $1$ and $-1$ in $\IH^\times$ maps to $Id \in SO(3)$, the kernel is not trivial. So it is not a isomorphism.

(viii) Notice that the kernel of the homomorphism is actually $\{1,-1\}\subset \IH^\times$. Since this is discrete, the preimage of any open set $V\in SO(3)$ is disjoint in $\IH^\times \isom SU(2)$. Therefore it is a covering map. In fact $SU(2)$ being diffeomorphic to $S^3$ and the kernel being isomorphic to $\IZ/2\IZ$ implies $SO(3)$ is diffeomorphic to $\IR P^3$. \\
\end{answer*}

\section*{Lie Algebras}
\label{sec:Lie Algebras}

\begin{problem*} 4.4
\end{problem*}

\begin{answer*}\hfill

(i) Let $X= \sum_{i=1}^n a_i \frac{\p}{\p x_i}$ and $Y = \sum_{j=1}^n b_j \frac{\p}{\p x_j}$ where $a_i$ and $b_j$ are real numbers. Then for $h\in \sC^\infty(\IR^n)$,
\[[X,Y]h = \sum_{i,j} (a_i \frac{\p b_j}{\p x_i}\frac{\p}{\p x_j}  - b_j \frac{\p a_i}{\p x_j}\frac{\p}{\p x_i})(h) = 0\]
Therefore it is a Lie subalgebra.

(ii) By (i), it is abelian.

(iii) Yes. For $X= \sum_{i=1}^n a_i \frac{\p}{\p x_i}$ and $Y = \sum_{j=1}^n f_j \frac{\p}{\p x_j}$ where $a_i\in IR$ and $f_j\in \sC^\infty(\IR^n)$. Then,
\[[X,Y] = \sum_{i,j} a_i\frac{\p f_j}{\p x_i}\frac{\p }{\p x_j} \in Der(\sC^\infty(\IR^n))\]
\end{answer*}

\end{document}
