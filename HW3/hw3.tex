\documentclass[10pt]{article}
\usepackage{amsmath}
\usepackage{amsthm}
\usepackage{amsfonts}
\usepackage{amssymb}
\usepackage{latexsym}
\usepackage{epsfig}
\usepackage{graphicx}

\usepackage[matrix,tips,graph,curve]{xy}

\newcommand{\mnote}[1]{${}^*$\marginpar{\footnotesize ${}^*$#1}}
\linespread{1.065}

\makeatletter

\setlength\@tempdima{5.5in}
\addtolength\@tempdima{-\textwidth}
\addtolength\hoffset{-0.5\@tempdima}
\setlength{\textwidth}{5.5in}
\setlength{\textheight}{8.75in}
\addtolength\voffset{-0.625in}

\makeatother

\makeatletter
\@addtoreset{equation}{section}
\makeatother


\renewcommand{\theequation}{\thesection.\arabic{equation}}

\theoremstyle{plain}
\newtheorem{theorem}[equation]{Theorem}
\newtheorem{corollary}[equation]{Corollary}
\newtheorem{lemma}[equation]{Lemma}
\newtheorem{proposition}[equation]{Proposition}
\newtheorem{conjecture}[equation]{Conjecture}
\newtheorem{problem}[equation]{Problem}
\newtheorem{fact}[equation]{Fact}
\newtheorem{facts}[equation]{Facts}
\newtheorem*{theoremA}{Theorem A}
\newtheorem*{theoremB}{Theorem B}
\newtheorem*{theoremC}{Theorem C}
\newtheorem*{theoremD}{Theorem D}
\newtheorem*{theoremE}{Theorem E}
\newtheorem*{theoremF}{Theorem F}
\newtheorem*{theoremG}{Theorem G}
\newtheorem*{theoremH}{Theorem H}

\theoremstyle{definition}
\newtheorem{definition}[equation]{Definition}
\newtheorem{definitions}[equation]{Definitions}
%\theoremstyle{remark}

\newtheorem{remark}[equation]{Remark}
\newtheorem{remarks}[equation]{Remarks}
\newtheorem{exercise}[equation]{Exercise}
\newtheorem*{exercise*}{Exercise}
\newtheorem{example}[equation]{Example}
\newtheorem{examples}[equation]{Examples}
\newtheorem{notation}[equation]{Notation}
\newtheorem{question}[equation]{Question}
\newtheorem{assumption}[equation]{Assumption}
\newtheorem*{claim}{Claim}
\newtheorem{answer}[equation]{Answer}


\newenvironment{problem_hw}[1]{{\noindent\bfseries Problem #1.}}{}
\newenvironment{answer_hw}{{\noindent\em Answer.}}{}

%%%%%% letters %%%%
\newcommand{\fA}{\mathfrak{A}}
\newcommand{\fB}{\mathfrak{B}}
\newcommand{\fC}{\mathfrak{C}}
\newcommand{\fD}{\mathfrak{D}}
\newcommand{\fE}{\mathfrak{E}}
\newcommand{\fF}{\mathfrak{F}}
\newcommand{\fG}{\mathfrak{G}}
\newcommand{\fH}{\mathfrak{H}}
\newcommand{\fI}{\mathfrak{I}}
\newcommand{\fJ}{\mathfrak{J}}
\newcommand{\fK}{\mathfrak{K}}
\newcommand{\fL}{\mathfrak{L}}
\newcommand{\fM}{\mathfrak{M}}
\newcommand{\fN}{\mathfrak{N}}
\newcommand{\fO}{\mathfrak{O}}
\newcommand{\fP}{\mathfrak{P}}
\newcommand{\fQ}{\mathfrak{Q}}
\newcommand{\fR}{\mathfrak{R}}
\newcommand{\fS}{\mathfrak{S}}
\newcommand{\fT}{\mathfrak{T}}
\newcommand{\fU}{\mathfrak{U}}
\newcommand{\fV}{\mathfrak{V}}
\newcommand{\fW}{\mathfrak{W}}
\newcommand{\fX}{\mathfrak{X}}
\newcommand{\fY}{\mathfrak{Y}}
\newcommand{\fZ}{\mathfrak{Z}}
%%%%%%%%%%%%%%%%%%%%%%%%%%%%%%
\newcommand{\fa}{\mathfrak{a}}
\newcommand{\fb}{\mathfrak{b}}
\newcommand{\fc}{\mathfrak{c}}
\newcommand{\fd}{\mathfrak{d}}
\newcommand{\fe}{\mathfrak{e}}
\newcommand{\ff}{\mathfrak{f}}
\newcommand{\fg}{\mathfrak{g}}
\newcommand{\fh}{\mathfrak{h}}
\newcommand{\ffi}{\mathfrak{i}}  %% \fi is defined
\newcommand{\fj}{\mathfrak{j}}
\newcommand{\fk}{\mathfrak{k}}
\newcommand{\fl}{\mathfrak{l}}
\newcommand{\fm}{\mathfrak{m}}
\newcommand{\fn}{\mathfrak{n}}
\newcommand{\fo}{\mathfrak{o}}
\newcommand{\fp}{\mathfrak{p}}
\newcommand{\fq}{\mathfrak{q}}
\newcommand{\fr}{\mathfrak{r}}
\newcommand{\fs}{\mathfrak{s}}
\newcommand{\ft}{\mathfrak{t}}
\newcommand{\fu}{\mathfrak{u}}
\newcommand{\fv}{\mathfrak{v}}
\newcommand{\fw}{\mathfrak{w}}
\newcommand{\fx}{\mathfrak{x}}
\newcommand{\fy}{\mathfrak{y}}
\newcommand{\fz}{\mathfrak{z}}
%%%%%%%%%%%%%%%%%%%%%%%%%%%%%%%
\newcommand{\sA}{\mathcal{A}\,}
\newcommand{\sB}{\mathcal{B}\,}
\newcommand{\sC}{\mathcal{C}}
\newcommand{\sD}{\mathcal{D}\,}
\newcommand{\sE}{\mathcal{E}\,}
\newcommand{\sF}{\mathcal{F}\,}
\newcommand{\sG}{\mathcal{G}\,}
\newcommand{\sH}{\mathcal{H}}
\newcommand{\sI}{\mathcal{I}\,}
\newcommand{\sJ}{\mathcal{J}\,}
\newcommand{\sK}{\mathcal{K}\,}
\newcommand{\sL}{\mathcal{L}\,}
\newcommand{\sM}{\mathcal{M}\,}
\newcommand{\sN}{\mathcal{N}}
\newcommand{\sO}{\mathcal{O}}
\newcommand{\sP}{\mathcal{P}\,}
\newcommand{\sQ}{\mathcal{Q}\,}
\newcommand{\sR}{\mathcal{R}}
\newcommand{\sS}{\mathcal{S}}
\newcommand{\sT}{\mathcal{T}\,}
\newcommand{\sU}{\mathcal{U}\,}
\newcommand{\sV}{\mathcal{V}\,}
\newcommand{\sW}{\mathcal{W}\,}
\newcommand{\sX}{\mathcal{X}\,}
\newcommand{\sY}{\mathcal{Y}\,}
\newcommand{\sZ}{\mathcal{Z}\,}
%%%%%%%%%%%%%%%%%%%%%%%%%%%%%%%
\newcommand{\IA}{\mathbb{A}}
\newcommand{\IB}{\mathbb{B}}
\newcommand{\IC}{\mathbb{C}}
\newcommand{\ID}{\mathbb{D}}
\newcommand{\IE}{\mathbb{E}}
\newcommand{\IF}{\mathbb{F}}
\newcommand{\IG}{\mathbb{G}}
\newcommand{\IH}{\mathbb{H}}
\newcommand{\II}{\mathbb{I}}
\newcommand{\IK}{\mathbb{K}}
\newcommand{\IL}{\mathbb{L}}
\newcommand{\IM}{\mathbb{M}}
\newcommand{\IN}{\mathbb{N}}
\newcommand{\IO}{\mathbb{O}}
\newcommand{\IP}{\mathbb{P}}
\newcommand{\IQ}{\mathbb{Q}}
\newcommand{\IR}{\mathbb{R}}
\newcommand{\IS}{\mathbb{S}}
\newcommand{\IT}{\mathbb{T}}
\newcommand{\IU}{\mathbb{U}}
\newcommand{\IV}{\mathbb{V}}
\newcommand{\IW}{\mathbb{W}}
\newcommand{\IX}{\mathbb{X}}
\newcommand{\IY}{\mathbb{Y}}
\newcommand{\IZ}{\mathbb{Z}}
%%%%%%%%%%%%%%%%%%%%%%%%%%%%%%
\newcommand{\tA}{\mathrm {A}}
\newcommand{\tB}{\mathrm {B}}
\newcommand{\tC}{\mathrm {C}}
\newcommand{\tD}{\mathrm {D}}
\newcommand{\tE}{\mathrm {E}}
\newcommand{\tF}{\mathrm {F}}
\newcommand{\tG}{\mathrm {G}}
\newcommand{\tH}{\mathrm {H}}
\newcommand{\tI}{\mathrm {I}}
\newcommand{\tJ}{\mathrm {J}}
\newcommand{\tK}{\mathrm {K}}
\newcommand{\tL}{\mathrm {L}}
\newcommand{\tM}{\mathrm {M}}
\newcommand{\tN}{\mathrm {N}}
\newcommand{\tO}{\mathrm {O}}
\newcommand{\tP}{\mathrm {P}}
\newcommand{\tQ}{\mathrm {Q}}
\newcommand{\tR}{\mathrm {R}}
\newcommand{\tS}{\mathrm {S}}
\newcommand{\tT}{\mathrm {T}}
\newcommand{\tU}{\mathrm {U}}
\newcommand{\tV}{\mathrm {V}}
\newcommand{\tW}{\mathrm {W}}
\newcommand{\tX}{\mathrm {X}}
\newcommand{\tY}{\mathrm {Y}}
\newcommand{\tZ}{\mathrm {Z}}
%%%%%%% macros %%%%%

%% my definitions %%%

\newcommand{\End}{\mathrm{End}}
\newcommand{\tr}{\mathrm{tr}}
%\newcommand{\ind}{\mathrm{ind}}

\renewcommand{\index}{\mathrm{index \,}}
\newcommand{\Hom}{\mathrm{Hom}}
\newcommand{\Aut}{\mathrm{Aut}}
\newcommand{\Trace}{\mathrm{Trace}\,}
\newcommand{\Res}{\mathrm{Res}\,}
\newcommand{\rank}{\mathrm{rank}}
%\renewcommand{\dim}{\mathrm{dim}}

\renewcommand{\deg}{\mathrm{deg}}
\newcommand{\spin}{\rm Spin}
\newcommand{\Spin}{\rm Spin}
\newcommand{\erfc}{\rm erfc\,}
\newcommand{\sgn}{\rm sgn\,}
\newcommand{\Spec}{\rm Spec\,}
\newcommand{\diag}{\rm diag\,}
\newcommand{\Fix}{\mathrm{Fix}}
\newcommand{\Ker}{\mathrm{Ker \,}}
\newcommand{\Coker}{\mathrm{Coker \,}}
\newcommand{\Sym}{\mathrm{Sym \,}}
\newcommand{\Hess}{\mathrm{Hess \,}}
\newcommand{\grad}{\mathrm{grad \,}}
\newcommand{\Center}{\mathrm{Center}}
\newcommand{\Lie}{\mathrm{Lie}}
\newcommand{\ch}{\rm ch} % Chern Character
\newcommand{\rk}{\rm rk}
\newcommand{\sign}{\rm sign}
\renewcommand\dim{{\rm dim\,}}
\renewcommand\det{{\rm det\,}}
\newcommand{\dimKrull}{{\rm Krulldim\,}}
\newcommand\Rep{\mathrm{Rep}}
\newcommand\Hilb{\mathrm{Hilb}}
\newcommand\vol{\mathrm{vol}}
\newcommand\QED{\hfill $\Box$} %{\bf QED}}
\newcommand\Pf{\nonintend{\em Proof. }}
\newcommand\reals{{\mathbb R}}
\newcommand\complexes{{\mathbb C}}
\renewcommand\i{\sqrt{-1}}
\renewcommand\Re{\mathrm Re}
\renewcommand\Im{\mathrm Im}
\newcommand\integers{{\mathbb Z}}
\newcommand\quaternions{{\mathbb H}}
\newcommand\iso{{\cong}}
\newcommand\tensor{{\otimes}}
\newcommand\Tensor{{\bigotimes}}
\newcommand\union{\bigcup}
\newcommand\onehalf{\frac{1}{2}}
%\newcommand\Sym[1]{{Sym^{#1}(\complexes^2)}}
\newcommand\lie[1]{{\mathfrak #1}}
\renewcommand\fk{\mathfrak{K}}
\newcommand\smooth{\mathcal{C}^{\infty}}
\newcommand\trivial{{\mathbb I}}
\newcommand\widebar{\overline}

%%%%%Delimiters%%%%

\newcommand{\<}{\langle}
\renewcommand{\>}{\rangle}

%\renewcommand{\(}{\left(}
%\renewcommand{\)}{\right)}


%%%% Different kind of derivatives %%%%%
\newcommand{\delbar}{\bar{\partial}}
\newcommand{\pdu}{\frac{\partial}{\partial u}}
%\newcommand{\pd}[1][2]{\frac{\partial #1}{\partial #2}}

%%%%% Arrows %%%%%
%\renewcommand{\ra}{\rightarrow}                   % right arrow
%\newcommand{\lra}{\longrightarrow}              % long right arrow
%\renewcommand{\la}{\leftarrow}                    % left arrow
%\newcommand{\lla}{\longleftarrow}               % long left arrow
%\newcommand{\ua}{\uparrow}                     % long up arrow
%\newcommand{\na}{\nearrow}                      %  NE arrow
%\newcommand{\llra}[1]{\stackrel{#1}{\lra}}      % labeled long right arrow
%\newcommand{\llla}[1]{\stackrel{#1}{\lla}}      % labeled long left arrow
%\newcommand{\lua}[1]{\stackrel{#1}{\ua}}      % labeled  up arrow
%\newcommand{\lna}[1]{\stackrel{#1}{\na}}      % labeled long NE arrow

\newcommand{\into}{\hookrightarrow}
\newcommand{\tto}{\longmapsto}
\def\llra{\longleftrightarrow}

\def\d/{/\mspace{-6.0mu}/}
\newcommand{\git}[3]{#1\d/_{\mspace{-4.0mu}#2}#3}
\newcommand\zetahilb{\zeta_{{\text{Hilb}}}}
\def\Fy{\sF \mspace{-3.0mu} \cdot \mspace{-3.0mu} y}
\def\tv{\tilde{v}}
\def\tw{\tilde{w}}
\def\wt{\widetilde}
\def\wtilde{\widetilde}
\def\what{\widehat}

%%%%%%%%%%%%%%%%%%% Mark's definitions %%%%%%%%%%%%%%%%%%%%

\newcommand{\frakg}{\mbox{\frakturfont g}}
\newcommand{\frakk}{\mbox{\frakturfont k}}
\newcommand{\frakp}{\mbox{\frakturfont p}}
\newcommand{\q}{\mbox{\frakturfont q}}
\newcommand{\frakn}{\mbox{\frakturfont n}}
\newcommand{\frakv}{\mbox{\frakturfont v}}
\newcommand{\fraku}{\mbox{\frakturfont u}}
\newcommand{\frakh}{\mbox{\frakturfont h}}
\newcommand{\frakm}{\mbox{\frakturfont m}}
\newcommand{\frakt}{\mbox{\frakturfont t}}
\newcommand{\G}{\Gamma}
\newcommand{\g}{\gamma}
\newcommand{\fraka}{\mbox{\frakturfont a}}
\newcommand{\db}{\bar{\partial}}
\newcommand{\dbs}{\bar{\partial}^*}
\newcommand{\p}{\partial}

%%%%%%%%%%%%% new definitions for the positive mass paper %%%%%%%%%

\newcommand{\sperp}{{\scriptscriptstyle \perp}}

%%%%%%%%%%%%%%%%%%%%%%%

%%%%%%%%%%%%%%%%%%%%%%%%%%%%%%%%%%%%%%%%%%%%%



%
\begin{document}
%
\title{Math 603 -- Homework 3}
\author{Feng Gui}
\date{\today}
\maketitle

\section*{Exercise Sheet 1}
\label{sec:Exercise Sheet 1}

\begin{problem_hw}{1 (i)}
\end{problem_hw}

\begin{answer_hw}

Yes. Consider the diagonal elements of $N^k$. Name the $k$-diagonal to be the entries $a_ij$ with $j-i = k$ where $\{a_ij\}$ is the matrix. So the main diagonal is 0-th diagonal. Notice only 1-diagonal of $N$ has non-zero elemnents. Using mathematical induction, we can prove that only the k-diagonal $N^k$ has non-zero elements. Therefore $N^n$ must be the zero matrix. Thus $N$ is nilpotent. \\
\end{answer_hw}

\begin{problem_hw}{1 (ii)}
\end{problem_hw}

\begin{answer_hw}

No. Consider $D$ has $1,2,3$ on its main diagonal. $N$ has $1,1$ on its 1-diagonal. Then $ND \not = DN$.\\
\end{answer_hw}

\begin{problem_hw}{1 (iii)}
\end{problem_hw}

\begin{answer_hw}
\[\exp(D) = diag(e^{d_1},e^{d_2},...,e^{d_n})\]
\end{answer_hw}


\begin{problem_hw}{1 (iv)}
\end{problem_hw}

\begin{answer_hw}
\[\exp(J) = \exp(D)\sum_{k = 0}^{n-1} \frac{1}{k!}N^k\]
where $0!$ is 0 and $N^0 = I_n$ \\
\end{answer_hw}


\begin{problem_hw}{2 (i)}
\end{problem_hw}

\begin{answer_hw}
I did.\\
\end{answer_hw}

\begin{problem_hw}{2 (ii)}
\end{problem_hw}

\begin{proof} \hfill

First we show that $L$ is a vector space over $\IR$. This is quite obvious as scalar multiplication and matrix addition keep the lower triangle and the main diagonal to be 0. Notice that the matrix multiplication can keep those entries to be 0 as well.

Now we show that the Lie bracket $[X,Y] = XY-YX$ for $X,Y\in L$ is bilinear. Since $[aX+bY, Z] = (aX+bY)Z - Z(aX+bY) = a[X,Z] + b[Y,Z]$ for any $X,Y,Z\in L$ and $a,b \in \IR$, the Lie bracket is linear in the first entry. Similarly, it is linear for the second entry. Therefore $L$ forms a Lie algebra.

\end{proof}

\begin{problem_hw}{2 (iii)}
\end{problem_hw}

\begin{answer_hw}

Notice that for $X,Y\in L$, $[X,Y]$ has only one entry that is not equal to 0. Therefore the derived Lie subgroup is the $N = \{\{x_{ij}\}\in M_3(\IR)\vert x_{ij} = 0\, \mathrm{for}\, j-i <2\}$, that is, only the entry on the first row third column is non-zero. \\

\end{answer_hw}

\begin{problem_hw}{2 (iv)}
\end{problem_hw}

\begin{answer_hw}

Clearly, one dimensional vector spaces over a field $k$ are isomorphic to each other. However, for any $X, Y\in L$ and $X\not = 0$ where $L$ is an one dimensional Lie algebra, we have $[X,Y] = [X, cX] = 0$. So the one dimensional Lie algebra has trivial extra structure on vector space. Therefore up to isomorphism there's only one Lie algebra over field $k$. \\
\end{answer_hw}

\begin{problem_hw}{2 (v)}
\end{problem_hw}

\begin{answer_hw}

$L'$ is isomorphic to $\IR$.\\

\end{answer_hw}

\begin{problem_hw}{2 (vi)}
\end{problem_hw}

\begin{answer_hw}

$L/L'$ is a two dimensional vector space which is also a abelian Lie algebra. So it is isomorphic to $\IR^2$, where the Lie bracket of $\IR^2$ is the natural construction from multiplication of real numbers, $[(a,b), (c,d)]= ([a,c],[b,d]) = (ac-ca,bd-db)$. \\

\end{answer_hw}

\begin{problem_hw}{2 (vii)}
\end{problem_hw}

\begin{answer_hw}

The element of $H$ has $(1,1,1)$ as the main diagonal, zeros on lower triangle and any number on upper triangle.\\

\end{answer_hw}

\begin{problem_hw}{2 (viii)}
\end{problem_hw}

\begin{answer_hw}

$H' = L'$, the derived Lie subalgebra of $L$. \\

\end{answer_hw}

\begin{problem_hw}{2 (ix)}
\end{problem_hw}

\begin{answer_hw}
$H' \cong \IR$ \\

\end{answer_hw}

\begin{problem_hw}{2 (x)}
\end{problem_hw}

\begin{answer_hw}
$H/H' \cong \IR^2$ \\

\end{answer_hw}


\begin{problem_hw}{3}
\end{problem_hw}

\begin{answer_hw}

Suppose $X = gYg^{-1}$ with $Y$ being a upper triangle matrix. Then $$\det(e^X) = \det(e^{gYg^{-1}}) = \det(e^Y) = e^\mathrm{Tr}(Y) = e^\mathrm{Tr}(X)$$

Therefore if $X$ has zero trace, the determinant of its exponential is 1.\\

\end{answer_hw}


\section*{Lie Groups: Beyond an Introduction}
\label{sec:Lie Groups: Beyond an Introduction}

\begin{problem_hw}{1.1}
\end{problem_hw}

\begin{answer_hw}

Since $\det(e^X) = e^{\mathrm{Tr}(X)}$, the exponential maps squate matrices to invertible matrices. To show it maps onto $GL_n(\IC)$, we let $X = gYg^{-1}$ where $Y$ is in Jordan normal form. It is not hard to construct a $Z\in \fg\fl(n,\IC)$ such that $e^Z = Y$. Then we have $e^{gZg^{-1}} = ge^Zg^{-1} = gYg^{-1} = X$. \\

\end{answer_hw}

\begin{problem_hw}{1.2}
\end{problem_hw}

\begin{answer_hw}

Let $$c(t) = \begin{pmatrix} a & z(t) \\ 0 & a^{-1} \end{pmatrix}$$

Since $z'(0)$ can be any complex number, $\{c'(0)\} = G$ but with a Lie algebra structure, i.e. $$ L = \left\{X\in\fg\fl(2,\IC)\left| X = \begin{pmatrix} a & z \\ 0 & a^{-1} \end{pmatrix} \right.\right\}$$

\end{answer_hw}

\section*{Coutation of Tangent Spaces of Some Closed Subgroups of $GL_n(\IR)$}


\begin{problem_hw}{3.5}
\end{problem_hw}

\begin{proof}

We have $0 = f(Id+tY) = (Id + t\bar{Y}^T)(Id+tY)-Id = t(\bar{Y}^T+Y)$ as $t^2 = 0$. So the group locally are skew hermitian matrices that lies in $\IR^{4n^2}$ space. The group has rank $4\cdot\frac{n(n-1)}{2}+3n = 2n^2+n$. Since $\bar{X}^T X - Id$ is always hermitian (symmetric), The image of $f$ has only $4\cdot\frac{n(n-1)}{2}+n = 2n^2-n$ rank. Therefore by implicit function theorem, the $Sp(2n)$ is locally the space of skew hermitian matrices, which forms a real subspace of $M_n(\IH)$\\

\end{proof}

 \end{document}
